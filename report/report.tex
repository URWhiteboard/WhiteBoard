\documentclass{article}

\usepackage{graphicx}
\usepackage[margin=1 in]{geometry}

\DeclareGraphicsExtensions{.png,.jpg}

\begin{document}

\begin{center}

    \huge{\textbf{WhiteBoard}}

    \huge{\textit{An Efficient and Intuitive Learning Management System}}

    \huge{Project Progress Report}

    \vspace{10 pt}

    \large{
        \begin{tabular}{cccc}
            Noah Christiano&Joel Kottas&Shir Maimon&Jacob Roschen\\
        \end{tabular}
    }

\end{center}

\vspace{10 pt}

\section{Background}

A Learning Management System(LMS) is an interface between students, professors,
and school administration. One of the most popular LMSes, especially at
colleges and universities, is Blackboard\cite{Blackboard}. Blackboard, however,
is extremely slow and has unnecessary features that are often left unused. For
example, Blackboard's personalize page features are rarely used, and students
avoid the discussion boards. These LMSes' designs often make them difficult to
navigate. For example, Blackboard has two sets of navigation tabs, and it is
not obvious what the difference between them is. These problems add up to
create an interface that is slow and difficult to use, even though there is no
need for it to be that way. Students today would rather get what they need
quickly than have hundreds of features that they rarely, if ever, use. Other
LMSes include Moodle\cite{Moodle}, Veracross\cite{Veracross}, and
Canvas\cite{Canvas}. Moodle is open-source, but it is difficult to use.
Veracross is nice, but expensive. Canvas is more like Blackboard and is just
not good.

\section{Objective}

WhiteBoard will focus on having an efficient, intuitive interface, and it will
have features of excellent quality rather than a large quantity of features. It
is open-source, allowing it to be customized and used as desired, although it
focuses on the University of Rochester. Also, WhiteBoard will have a unified
user experience. It shouldn't take a student who is majoring in computer
science three days to find where the grades are posted, and even then to be
confused about what their grade is, and it should be easy to navigate to the
University of Rochester website. WhiteBoard's design will make such navigation
a pleasure.

\section{Features}

Specifically, WhiteBoard will have the following features:
\begin{itemize}
    \item Lightning-quick load times
    \item A secure and painless login system
    \item Course system, with registration and grades for students
    \item Convenient communication between administrators, professors, and
        students, both for general communication and announcements
    \item Open source
\end{itemize}

\section{Current Assets}

Our team's combined experience covers PHP and Javascript very well, the LAMP
stack and all its associated programming languages, and even configuring a
Linux server for web hosting. Everyone on the team understands design, and how
to learn from both the successes and the mistakes of systems like Blackboard.
As students, we have constant access to Blackboard, and we have the expertise
to dive into the site to see what code works well and what code doesn't. We are
also constantly learning, and our ability in areas relating to this project
will only increase.

We are also using several libraries to assist in creating WhiteBoard. We are
using jQuery, a javascript library, and code from the open source PHP-Login
project, so that we have a secure but simple to use system. The PHP-Login
library also includes a module for sending email, so we don't need to deal with
the intricacies of that. There are also many other libraries for Javascript and
PHP available should we find them useful.

The server we are using is an Apache server on Ubuntu.

We're also using PHPMyAdmin and MySQL Workbench for designing and managing the
database.

\section{Budget}

A large budget shouldn't be necessary. WhiteBoard requires server space on
which it can be hosted, but we are planning to get server space on the computer
science servers to save money. If we can't get server space there, then we
shouldn't need to spend more than \$10 per month on hosting.

\section{Schedule and Plan}

\subsection{Development Environment}

We are collaborating over GitHub, which combines version control, file sharing,
and the ability to edit files online into one service.

We will rent a dedicated server from Online hosting. This Ubuntu server hosts a
LAMP stack which is the foundation for the website. The web server hosts two
copies of the website, one with a functioning version of the site and another
with a test version of the site. This allows us to push changes to the server
via GitHub and test our code in real time. The database will use MySQL and the
SQL programming language. Our database structure will be similar to the open
source moodle project's. However, ours will be greatly simplified in order to
promote efficiency.

\subsection{Database Structure}

We have thoroughly planned the database for the entire system. Fundamentally,
the database dictates how the PHP code is organized and implemented. The
relationships follow fairly closely with how the back-end will be implemented
and integrated. A diagram of the database structure follows.

\includegraphics[width=6.5 in]{db}

\subsection{Login System}

With the open-source library we're using, the PHP-Login project, most of the
login system is complete. Some adjustments for our front-end and our database
structure may be necessary, but they should be fairly small.

\subsection{General Layout}

The general layout of the page is straightforward: there is a header, sidebar,
and the main content area. The goal of this project is not to make anything too
fancy, and keeping the user interface simple like this is one of the main ways
we accomplish this goal. Below is an image of the current general layout.

\includegraphics[width=6.5 in]{general_layout}

\subsection{Courses and Grades}

Courses will be primarily a way of organizing students with their teachers and
their grades. Grades and announcements/communication will go through courses so
that they go to the right people. Grades are some of the most important data of
the system, but they are fairly simple data. Scaling gets a bit complicated -
we plan on implementing it, but this can be removed if necessary for time.

\subsection{Announcements and Communication}

The primary aim of announcements and communication is for students to be able
to contact other students in their classes. For simplicity, this directs to
email. It also allows teachers to get a list of their students for whole-class
announcements, and for students to get their teachers' emails easily.

\subsection{Presentation}

The system should have a clear visual UI and it should be simple enough that it
could even be explained with just screenshots and instructions. A the
motivation for using our system shouldn't even need a presentation, compared to
systems like Blackboard.

\bibliography{refs}{} \bibliographystyle{plain}

\end{document}
