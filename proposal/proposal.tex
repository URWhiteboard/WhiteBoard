\documentclass{article}

\usepackage[margin=1in]{geometry}

\begin{document}

\begin{center}
    \huge{\textbf{WhiteBoard}}
    
    \huge{\textit{An Efficient and Intuitive Learning Management System}}

    \huge{Project Proposal}

    \vspace{10 pt}

    \large{
        \begin{tabular}{cccc}
            Noah Christiano&Joel Kottas&Shir Maimon&Jacob Roschen\\
        \end{tabular}
    }

\end{center}

\vspace{10 pt}

\section{Background}

A Learning Management System(LMS) is an interface between students,
professors, and school administration. One of the most popular LMSes,
especially at colleges and universities, is Blackboard\cite{Blackboard}.
Blackboard, however, like other LMSes, is extremely slow and has unnecessary
features that are often left unused. For example, Blackboard’s personalize
page features are rarely used, and students avoid the discussion boards. These
LMSes’ designs also often make them difficult to navigate. For example,
Blackboard has two sets of navigation tabs, and it is not obvious what the
difference between them is. These problems add up to create an interface that
is slow and difficult to use, even though there is no need for it to be that
way. Students today would rather get what they need quickly than have hundreds
of features that they rarely, if ever, use. Other LMSes exist too, such as
Moodle\cite{Moodle}, Veracross\cite{Veracross}, and Canvas\cite{Canvas}, but
they have their own problems, and none of them are open-source or as
lightweight as we want.

\section{Objective}

WhiteBoard will focus on having an efficient, intuitive interface, and it will
have features of excellent quality rather than a large quantity of features. It
is open-source, allowing it to be customized and used as desired, it focuses on
the University of Rochester. Most importantly, WhiteBoard will have a unified
user experience. It shouldn't take a student who is majoring in computer
science three days to find where the grades are posted, and even then to be
confused about what their grade is, and it should be easy to navigate to the
University of Rochester website. WhiteBoard’s design will make such navigation
a pleasure.

\section{Features}

Specifically, WhiteBoard will have the following features:
\begin{itemize}
    \item Lightning-quick load times
    \item A secure and painless login system
    \item Course system, with registration and grades for students
        \begin{itemize}
            \item Possibly a bonus schedule optimizing utility
        \end{itemize}
    \item Calendar
    \item Convenient communication between administrators, professors, and
        students, both for general communication and announcements
    \item Open source
\end{itemize}

\section{Current Assets}

Our team’s combined experience covers PHP and Javascript very well, the
LAMP stack and all its associated programming languages, and even
configuring a Linux server for web hosting. Everyone on the team
understands design, and how to learn from both the successes and the
mistakes of systems like Blackboard. As students, we have constant access
to Blackboard, and we have the expertise to dive into the site to see what
code works well and what code doesn’t. We are also constantly learning, and
our ability in areas relating to this project will only increase.

We are also using several libraries to assist in creating WhiteBoard. We
are using jQuery, a javascript library, and code from the open source
PHP-Login project, so that we have a secure but simple to use system.

\section{Budget}
A large budget shouldn’t be necessary. WhiteBoard requires server space on
which it can be hosted, but we are planning to get server space on the computer
science servers to save money. If we can’t get server space there, then we
shouldn’t need to spend more than \$10 per month on hosting.

\section{Schedule and Plan}

In general, front-end programming/design (i.e. client-side) can be done in
parallel with back-end programming/design (i.e. server-side).

\vspace{10 pt}

\begin{tabular}{ll}
    Expected Date Done&Feature(s)\\
    10/23/14&Development environment and database structure\\
    10/25/14&Login system and general layout\\
    11/1/14&Courses and grades\\
    11/8/14&Announcements and communication\\
    11/15/14&Schedule/calendar\\
    11/29/14&Presentation\\
\end{tabular}

\subsection{Development Environment and database structure}

\subsection{Login system and general layout}

We will rent a dedicated server from Online hosting. This Ubuntu server hosts a LAMP stack which is the foundation for the website. The web server hosts two copies of the website, one with a functioning version of the site and another with a test version of the site. This allows us to push changes to the server via GitHub and test our code in real time.
The database will use MySQL and the SQL programming language. Our database structure will be similar to the open source moodle project's. However, ours will be greatly simplified in order to promote efficiency.

\bibliography{refs}{}
\bibliographystyle{plain}

\end{document}
